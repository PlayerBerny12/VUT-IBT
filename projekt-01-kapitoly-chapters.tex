\chapter{Úvod}

V době vysokorychlostního a stabilního internetu je využívání vzdálených uložišť běžnou záležitostí. Pro většinu uživatelů je to nástroj
pro jednoduchou synchronizaci dat napříč několika zařízeními od mobilního telefonu po stolní počítač. Pokud vezmeme v úvahu pouze velká datová centra,
jedná se pravděpodobně o nejlevnější a nejspolehlivější způsob uchování dat, protože tyto centra mají velkou kapacitu úložného prostoru a také zálohy na
softwarové i hardwarové úrovni. Pod pojmem vzdálené uložiště si nemusíme představit jen velká datová centra se stovkami serverů, může se jednat o relativně malé uložiště
ve firmě nebo dokonce o osobní/domácí řešení. Pro využívaní osobních/firemních uložišť se lidé uchylují v případech, je-li třeba uložit osobní nebo
určitým způsobem citlivá data, která by nemohla být poskytnuta třetí straně. Nabízená řešení také nemusí poskytovat potřebnou funkcionalitu nebo jejich finanční model
nesplňuje zákazníkova kritéria.

V průběhu let byly vyvinuty protokoly řešící sdílení souborů jako FTP, NFS a mnohé další. Sdílení souborů je komplexní záležitostí a obsahuje
velké množství parametrů. Každý protokol se zaměřuje jen na určité parametry jako zabezpečení a spolehlivost přenosu, oprávnění pro jednotlivé
uživatele a soubory, synchronizace modifikovaných souborů apod nebo pojme daný parametr odlišným způsobem. V dnešní době je většina komerčně využívaných
aplikací vystavěna na proprietárních protokolech nebo na protokolech umožňující obecnější použití. Jedním z nejpoužívanějších aplikačních protokolů
bude Hypertext Transfer Protokol neboli HTTP.

Cílem této práce je prozkoumat aktuální nabídku a možnosti na poli vzdálených uložišť a navrhnout aplikaci pro řízený přístup ke vzdáleným dokumentům pro
platformu \mbox{GNU/Linux}. Aplikace bude zaměřena na řízený životní cyklus souborů s vynucenými oprávněními pro jednotlivé soubory daná vzdáleným uložištěm.
Komunikace skrze HTTP protokol byla definovaná externím zadavatelem. Zdrojové kódy jsou dostupné na veřejném Github repozitáři.
\footnote{\url{https://github.com/PlayerBerny12/VUT-IBT-Code}}

\chapter{Motivace a existující řešení}

Popularita a využití vzdálených uložišť roste, protože lidé využívají více zařízení a potřebují mezi nimi jednoduchou synchronizaci nebo na jednom souboru
potřebuje spolupracovat více lidí. Dalším podnětem využívání vzdálených uložišť je většího množství dat a potřeba tyto data efektivně sdílet a bezpečně uložit.
Postupně se digitalizují další systémy například ve státní správě nebo dalších podnikatelských sektorech. 

Na trhu je velké množství služeb zaměřujících se na různé klientské potřeby. Diverzita nabídky je až překvapivě velká. Většina se zaměřuje na ukládání a sdílení
dat ve svém ekosystému, které jsou určené pro široké použití. Pokud jsou požadavky specifické, tak možnost vlastní konfigurace není možná. Pro částečnou úpravu 
nebo automatizaci procesů lze využít poskytované API, ve většině případu se jedná o variantu REST API. Poslední možností je vystavění obdobné služby z několika
existujících aplikací nebo vytvoření nové.\\

\noindent V kontextu této práce jsou požadavky následující: 

\begin{itemize}
    \item Dodržování oprávnění i na klientském systému
    \begin{itemize}
        \item read-only
        \item write-only
        \item read/write
    \end{itemize}
    \item Životní cyklus souboru (stažení)
    \begin{itemize}
        \item stažení
        \item případná modifikace
        \item smazání souboru po vypršení platnosti
    \end{itemize}
    \item Autentifikace
    \begin{itemize}
        \item uživatelské jméno a klientský certifikát
        \item uživatelské jméno a heslo
    \end{itemize}
\end{itemize}

\section{Současná řešení}

Pro bližší analýzu byl vybrán vzorek služeb a programů obsahující velké ekosystémy služeb po open source aplikace. Zkoumáno bylo několik parametrů,
jako poskytovaná funkcionalita pro jednotlivce a firmy, cena, integrace s ostatními systémy a aplikacemi apod. Obecně nelze jednoznačně určit nejlepší uložiště, 
ale je možné poukázat na rozdílné vlastnosti a vyzdvihnout kladné. Koncový uživatel se následně může rozhodnout dle svých potřeb.

\section{Google Drive}

Pro synchronizaci obsahu z Google Drive na osobní počítač se využívá aplikace pojmenovaná Google File Stream. Je možné ji provozovat pouze na systémech
Windows a Mac OS.\cite{GoogleFileStream} Pro Linuxové distribuce lze využít například neoficiální open source aplikaci 
Google Drive OCamlFUSE\footnote{\url{https://github.com/astrada/google-drive-ocamlfuse}} využívající technologii
FUSE, která bude popsána v následující kapitole. Google Disk poskytuje plně funkční webové rozhraní,
ve kterém je možné dokumenty přímo upravovat bez nutnosti stahování. Problém nenastal ani v případě konkurenčních Microsoft Office dokumentů. 

Google Workspace (dříve Google Suite) je možné pořídit v několika balíčcích. Například balíček Bussiness Standard nabízí 2 TB uložiště za 10,40 EUR měsíčně.
Balíček obsahuje další služby jako firemní email a schůzky až o 150 účastnících na Google Meet.\cite{GoogleWorkspace}

Každý uživatel má vlastní disk s omezenou kapacitou podle balíčku. Na disku lze vytvářet standartní složkovou hierarchii a je možné sdílet jednotlivé soubory nebo
obsah složek s ostatními Google uživateli. Verze Enterprise umožňuje vytváření dalších disků, na které je možné přiřadit seznam uživatelů. Každý uživatel
má nastavenou jednu z 6 rolí, které vymezují jeho možnosti. K dispozici je API, která pokrývá veškeré možnosti webového rozhraní jako vytvoření souboru,
sdílení atd.\cite{GoogleAPIReference}

\section{Dropbox}

Dropbox má oficiální balíčky své aplikace pro Ubuntu a Fedoru, případně je možné si aplikaci zkompilovat ze zdrojových souborů. Podpora Windows a Mac OS je samozřejmostí.
Z práce „Personal Cloud Storage Benchmarks and Comparison“ lze vyčíst, že Dropbox se zaměřuje na efektivitu přenosu a minimální zatížení sítě.
Využívá menší množství TCP spojení oproti jeho konkurentům a také stejně jako Google Drive komprimuje data před odesláním.\cite{CloudStorageComparison}
Toto řešení nemusí dosahovat nejvyšší výkonosti, ale nezatěžuje tolik infrastrukturu. 

Webové rozhraní je více orientováno na správu souborů a jejich historii. Nabízí jednoduché obnovení smazaných souborů nebo návrat k předchozí verzi.
Nenabízí tolik možností jako Google Drive, ale upravovat dokumenty ve webovém prohlížeči lze také. Na výběr je mezi Microsoft Office Online nebo Google Worksapce.
Sdílení souborů a složek je možné i s uživateli, kteří nemají Dropbox účet.

Balíček Plus za 9,99 EUR měsíčně dává uživateli přístup ke 2 TB uložišti. Firemní balíčky obsahují rozšíření pro lepší správu uživatelů, vytváření skupin
a admin panel/konzole. Dropbox má také API nabízející dostatečnou funkcionalitu pro případnou integraci s jinými systémy apod.\cite{Dropbox}

\section{pCloud}

Poskytovatel pCloud podporuje nejpoužívanější platformy, a to včetně mobilních. Za 9,99 EUR měsíčně dostane uživatel 2 TB uložiště. Jako jediný z poskytovatelů
v analýze nabízí balíček s jednorázovou platbou za cenu 350 EUR. Jedná se o identický balíček, pouze forma platby se liší. Všechna data jsou přenášena přes
šifrovaný kanál a všechny kopie souborů na pěti různých serverech jsou šifrovaná 256bitovým AES klíčem.\cite{pCloud}

Webové rozhraní je jednoduché, avšak oproti konkurentům má méně funkcí. Lze zobrazit náhled dokumentů, ale upravovat je nelze. Každý soubor má tzv. revize 
a je možné obnovit obsah souboru na vybranou revizi. Revize jsou uchovány po dobu jednoho roku. Soubory je možné sdílet i s uživateli, kteří nemají účet u pCloudu. 

Pro firemní zákazníky je nabízený rozšířený management uživatelů a monitoring s logy aktivity jednotlivých uživatelů.

\section{Syncthing}

Open source aplikace Syncthing synchronizuje soubory peer-to-peer. Nejedná se o čistou peer-to-peer architekturu, protože jeden z uzlů může být
nastaven jako server a veškeré data budou synchronizována vůči tomuto uzlu. Syncthing je možné provozovat na většině dnešních systémů jako Windows, Linux, BSD a
Mac OS.\cite{Syncthing}

Aplikace poskytuje webové rozhraní, které je určené pouze pro nastavení parametrů jako discovery protokol pro objevování ostatních uzlů, jaké složky mají být synchronizovány
nebo kolik verzí jednotlivých souborů má být uchováváno a mnoho dalšího. Velká přizpůsobivost umožňuje upravit fungování aplikace vlastním potřebám, na druhou
stranu bude náročné udržovat systém s větším množstvím uživatelů vystavěný na této aplikaci. Jedná se tedy spíše o domácí řešení.

\chapter{Návrh řešení a použité technologie}

\section{Architektura}
\subsection{Případ užití}

\section{Technologie}
\subsection{GNU/Linux}
\subsection{C++/C}
\subsection{OpenAPI}
\subsection{HTTP}

\cite{RFC7231}
\cite{HTTP3}

\subsection{Mock server}

\cite{MockServer}

\subsection{Souborový systém v uživatelském prostoru (FUSE)}

\cite{FuseOrNotToFuse}
\cite{HardeningFUSE}

\subsection{DBus}
\subsection{Keyring}

\cite{Keyring}

\chapter{Popis implementace a nasazení aplikace}

- Seznam použitých knihoven

\section{Definice API}
\section{Mock server}
\section{Vyžívané cesty/soubory v Linuxové adresářové struktuře}
\subsection{Konfigurační soubor}
\subsection{SQLite databáze}
\section{Autentifikace}
\section{Práce se soubory}
\subsection{Otevření souboru v příslušné aplikaci}

\cite{xdg}

\section{Nasazení aplikace}
\subsection{DPKG} %APT
\subsection{RPM} %DNF/YUM
\subsection{Sestavení ze zdrojových souborů}

\chapter{Testovací sada}

- Testy pro funkce třídy API

\section{Google Test}

\chapter{Závěr}



