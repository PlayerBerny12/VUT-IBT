\chapter{Úvod}

V době vysokorychlostního a stabilního internetu je využívání vzdálených uložišť běžnou záležitostí. Pro většinu uživatelů je to nástroj
pro jednoduchou synchronizaci dat napříč několika zařízeními od mobilního telefonu po stolní počítač. Pokud vezmeme v úvahu pouze velká datová centra,
jedná se pravděpodobně o nejlevnější a nejspolehlivější způsob uchování dat, protože tyto centra mívají různé zálohy na softwarové i hardwarové úrovni.
Pod pojmem vzdálené uložiště si nemusíme představit jen velká datová centra se stovkami serverů, může se jednat o relativně malé uložiště
ve firmě nebo dokonce o osobní/domácí řešení. Pro využívaní osobních/firemních uložišť se lidé uchylují v případech, je-li třeba uložit osobní nebo
určitým způsobem citlivá data, která by nemohla být poskytnuta třetí straně. Nabízená řešení také nemusí poskytovat potřebnou funkcionalitu nebo jejich finanční model
nesplňuje zákazníkova kritéria.

V průběhu let byly vyvinuty protokoly řešící sdílení souborů jako FTP, TFTP, SFTP, NFS a další. Sdílení souborů je komplexní záležitostí a obsahuje
velké množství parametrů. Každý protokol se zaměřuje jen na určité parametry jako zabezpečení a spolehlivost přenosu, oprávnění pro jednotlivé
uživatele a soubory, synchronizace modifikovaných souborů apod. V dnešní době je většina komerčně využívaných aplikací vystavěna na proprietárních protokolech nebo
na protokolech umožňující obecnější použití. Jedním z obecně nejpoužívanějších aplikačních protokolů bude Hypertext Transfer Protokol neboli HTTP.

Cílem této práce je prozkoumat aktuální nabídku a možnosti na poli vzdálených uložišť a navrhnout aplikaci pro řízený přístup ke vzdáleným dokumentům pro
platformu \mbox{GNU/Linux}. Aplikace bude zaměřena na řízený životní cyklus souborů s vynucenými oprávněními pro jednotlivé soubory daná vzdáleným uložištěm.
Komunikace skrze HTTP protokol byla definovaná externím zadavatelem. Zdrojové kódy jsou dostupné na veřejném Github repozitáři.
\footnote{\url{https://github.com/PlayerBerny12/VUT-IBT-Code}}

\chapter{Motivace a existující řešení}

Využívaní vzdálených uložišť roste, protože lidé využívají více zařízení a potřebují mezi nimi jednoduchou synchronizaci nebo na jednom souboru potřebuje spolupracovat
více lidí. Dalším podnětem využívání vzdálených uložišť je generování většího množství dat, které jako lidé vytváříme. Postupně se digitalizují další systémy například 
ve státní správě nebo dalších podnikatelských sektorech. 

Na trhu je velké množství služeb zaměřujících se na různé klientské sektory s různými požadavky. Diverzita těchto služeb je až překvapivě velká. Většina se snaží
o obecně použitelné ukládání/sdílení dat, pokud jsou požadavky na tuto službu specifické, tak možnost úpravy konfigurace těchto služeb nebývá velká. Poté nezbývá než vystavět
obdobnou službu z existujících aplikací nebo vytvořit novou na existujících technologiích.\\

\noindent V kontextu této práce jsou požadavky následující: 

\begin{itemize}
    \item Dodržování oprávnění i na klientském systému
    \begin{itemize}
        \item read-only
        \item write-only
        \item read/write
    \end{itemize}
    \item Životní cyklus souboru (stažení)
    \begin{itemize}
        \item stažení
        \item případná modifikace
        \item smazání souboru po vypršení platnosti
    \end{itemize}
    \item Autentifikace
    \begin{itemize}
        \item uživatelské jméno a klientský certifikát
        \item uživatelské jméno a heslo
    \end{itemize}
\end{itemize}

\section{Současná řešení}

Pro bližší analýzu bylo vybráno jen několik služeb a programů. Zkoumáno bylo několik parametrů, jako poskytovaná funkcionalita pro jednotlivce a firmy, cena,
integrace s ostatními aplikacemi apod. Služby od Googlu a Microsoftu není možné pořídit jednotlivě, ale dodávají se v balíčcích obsahující další software
nebo služby. Obecně nelze vybrat nejlepší vzdálené uložiště, ale je možné poukázat na rozdílné vlastnosti a koncový uživatel ať si vybere podle svých potřeb.

\section{Google File Stream}

Jedná se o apliakci synchronizující data ze všech Google disků, ke kterým má daný uživatel přístup. Aplikaci je možné provozovat pouze na systémech
Windows a Mac OS.\cite{GoogleFileStream} Google Disk poskytuje plně funkční webové rozhraní, ve kterém je možné dokumenty přímo upravovat bez nutnotsti stahování. 
Problém nenastane ani v případě Microsoft Office dokumentů, což je stále velmi využívaný softwareový balík kancelářských programů.

Google Workspace (dříve Google Suite) je možné pořídit v několika balíčcích. Například balíček Bussiness Standard nabízí 2 TB uložiště za 10,40 EUR. Balíček obashuje
další služby jako firemní email a schůzky až o 150 účastnících na Google Meet.\cite{GoogleWorkspace}

Každý uživatel má vlastní disk s omezenou kapacitou podle balíčku. Na disku lze vytvářet standartní složkovou hiearchii a je možné sdílet jednotlivé soubory nebo
obsah složek s ostatními uživateli. Verze Enterprise umožňuje vytváření dalších disků, na které je možné přiřadit seznam uživatelů. Každý uživatel má nastavenou jednu 
z 6 rolí, které vymezují jeho možnosti. K dispozici je API, která pokrývá veškeré možnosti webového rozhraní jako vytvoření souboru, sdílení apod.\cite{GoogleAPIReference}

\section{Dropbox}

Aplikace Dropbox má balíčky pro Ubuntu a Fedoru, případně jejich deriváty a také je možnost zkompilovat ji ze zdrojových souborů. 
Z práce "Personal Cloud Storage Benchmarks and Comparison" je vidět, že Dropbox se zaměřuje na efektivitu přenosu a minimální zatížení sítě.
Využívá manší množství TCP spojení oproti jeho konkurentům a také satejně jako Google Drive komprimuje data před odesláním.\cite{CloudStorageComparison}
Toto řešení nemusí dosahovat nejvyšší výkonosti, ale nezatěžuje tolik infrastrukturu. 

Webové rozhraní je více orientováno na správu souborů a jejich historii. Nabízí jednoduché obnovení smazaných souborů nebo návrat k předchozí verzi.
Nenabízí tolik možností jako Google Drive, ale upravovat dokumenty ve webovém prohlížeči lze také. Na výběr je mezi Microsoft Office Online nebo Google Worksapce.
Sdílení souborů a složek je možné i s uživateli, kteří nemají Dropbox účet.

Balíček Plus za 9,99 EUR dává uživateli přístup ke 2 TB uloložišti. Firemní balíčky obsahují rozšíření pro lepší správu uživatelů, vytváření skupin a admin panel/konzole.
Dropbox má také pro své uživatele API.\cite{Dropbox}

\section{Microsoft OneDrive}

\cite{Microsoft365}

\section{Syncthing}
\section{Rsync}

\chapter{Návrh řešení a použité technologie}

\section{Architektura}
\subsection{Případ užití}

\section{Technologie}
\subsection{GNU/Linux}
\subsection{C++/C}
\subsection{OpenAPI}
\subsection{HTTP}

\cite{RFC7231}
\cite{HTTP3}

\subsection{Mock server}

\cite{MockServer}

\subsection{Souborový systém v uživatelském prostoru (FUSE)}

\cite{FuseOrNotToFuse}
\cite{HardeningFUSE}

\subsection{DBus}
\subsection{Keyring}

\cite{Keyring}

\chapter{Popis implementace a nasazení aplikace}

- Seznam použitých knihoven

\section{Definice API}
\section{Mock server}
\section{Vyžívané cesty/soubory v Linuxové adresářové struktuře}
\subsection{Konfigurační soubor}
\subsection{SQLite databáze}
\section{Autentifikace}
\section{Práce se soubory}
\subsection{Otevření souboru v příslušné aplikaci}

\cite{xdg}

\section{Nasazení aplikace}
\subsection{DPKG} %APT
\subsection{RPM} %DNF/YUM
\subsection{Sestavení ze zdrojových souborů}

\chapter{Testovací sada}

- Testy pro funkce třídy API

\section{Google Test}

\chapter{Závěr}



